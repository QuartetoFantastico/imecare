\documentclass[12pt,a4paper]{report}
\usepackage[T1]{fontenc}
\usepackage[utf8]{inputenc}
\usepackage[brazilian]{babel}
\usepackage{amsmath}

\renewcommand\thesection{\arabic{section}}

\author{\textbf{
 Eduardo Delgado Bier
 Fernanda de Camargo Magano 
 Florence Alyssa Sakuma Shibata 
 Shayenne da Luz Moura 
 Théo Dury }}
\title{\textbf{Fase 1 - Documento de Requisitos do sistema}}
\date{outubro de 2015}

\begin{document}


% capa
\begin{titlepage}

\begin{center}
{\large IME-USP}\\[0.2cm]
{\large Trabalho de Engenharia de Software}\\[5.1cm]
{\bf \huge Fase 1 - Documento de Requisitos do sistema}\\[5.1cm]
\end{center}


\begin{large}

 Eduardo Delgado Bier \\
 Fernanda de Camargo Magano \\
 Florence Alyssa Sakuma Shibata \\
 Shayenne da Luz Moura \\
 Théo Dury \newline
 \end{large}


\begin{center}
{\large \textbf{São Paulo}}\\[0.2cm]
{\large \textbf{2015}} \\[0.2cm]
\end{center}

\end{titlepage}

\tableofcontents

\newpage

\section{Introdução}

\subsection{Objetivos deste software}

Quando uma pessoa decide tratar algum problema de saúde ou algo similar em um hospital, geralmente seus dados ficam restritos somente àquela unidade ou à sua rede hospitalar. Caso o hospital reconheça que o tratamento em sua unidade não é possível ou algum paciente tenha pretensões continuar seu tratamento em algum lugar, cabe ao paciente levar consigo suas informações a respeito do diagnóstico e o tratamento a que foi submetido.
Visando evitar o trabalho de transporte de informações de forma manual, o desperdício de tempo e de recursos, perda de informações e possíveis ambiguidades em suas interpretações, o sistema a ser desenvolvido neste projeto consiste na unificação de dados hospitalares em relação aos dados dos pacientes: ao diagnóstico, processo do tratamento, medicamentos indicados, estado atual do paciente, etc.

\textbf{Público Alvo:} esse software se destina principalmente aos médicos para terem acesso e poder modificar prontuários de pacientes e também é extretamente útil para os pacientes, pois podem acompanhar todo seu histórico de medicamentos, exames, consultas.



\subsection{Escopo do produto}

\subsubsection{Nome do produto e seus componentes}
IMECare:

\begin{itemize}
\item Organização de prontuários de pacientes
\item Fácil acesso às informações pelos médicos, enfermeiros e pacientes
\item Dados disponíveis na web, de modo que qualquer hospital integrado ao sistema tenha acesso
\item Garante praticidade e evita perda de informações importantes 
\end{itemize}


\subsubsection{Descrição do produto}

O produto será utilizado por hospitais e pacientes dos mesmos. Eles terão acesso a uma interface web (pretendemos utilizar um site) e por meio dela conseguirá acessar dados, modificar, checar os agendamentos e realizar alterações de datas, quando desejar.



\subsubsection{Missão do produto}

\begin{itemize}

\item Economizar tempo, já que com o histórico unificado dos pacientes, o médico já saberá qual é situação de todo o processo a que o paciente foi submetido;
\item Apresentar visão panorâmica do histórico e evolução do paciente;
\item Como consequência, pode-se utilizar os dados fornecidos por esse projeot para garantir um direcionamento na gestão de serviços de atendimento às pessoas doentes, permitindo organização por parte de quem trabalha nos hospitais. Assim, ocorre melhoria na qualidade do serviço e permite economia de tempo.

\end{itemize}




\subsection{Siglas e definições}

\textbf{SGBD} - Sistema Gerenciador de Banco de Dados \\
\textbf{RG}   - Registro Geral\\
\textbf{CPF}  - Cadastro de Pessoa Física\\



\section{Descrição geral do Produto}

\subsection{Perspectiva do produto} 

<Colocar o diagrama de caso de uso>

\subsection{Usuários do sistema}

\begin{itemize}
\item pacientes: podem acessar suas informações presentes no sistema e agendar/reagendar consultas
\item médicos: acessam/alteram os prontuários dos pacientes
\item enfermeiros: consultam informações dos prontuários dos pacientes
\item secretários: responsáveis por agendamento de consultas

\end{itemize}




\section{Requisitos específicos}


\subsection{Identificação dos requisitos}

\subsection{Prioridades dos requisitos}

Como o projeto é grande, é necessário ter um planejamento e analisar o que é mais prioritário dentre as tarefas que precisam ser realizadas. Abaixo, estão definidos o que significa ser: essencial, importante ou desejável. \\

\begin{itemize}
\item \textbf{Essencial:}
É o requisito sem o qual o sistema não entra em funcionamento.
São extremamente necessários, não podem ser adiados ou feitos parcialmente.

\item \textbf{Importante:}
Sem os requisitos importantes, o sistema até consegue entrar em funcionamento, mas não de forma satisfatporia. São requisitos que devem ser implementados, mas se não forem, ainda é possível ter o sistema em funcionamento.


\item \textbf{Desejável:}
É o requisito que não compromete as funcionalidades básicas do sistema. 
Podem ser deixados para versões posteriores do sistema, se não der tempo de ser concretizado na fase inicial.


\end{itemize}


\subsection{Descrição dos requisitos}

\begin{itemize}
\item \textbf{Gerenciamento dos pacientes}

O sistema precisa prover uma forma dos pacientes se cadastrarem no sistema e acessarem seus dados quando fizerem login.\\
\\ Os pacientes têm cpf, rg, contato (seja telefone/ endereço), tipo sanguíneo.
Um paciente pode ter uma ou mais doenças associadas a ele. Cada doença vai gerar um determinado grupo de exames pedidos. Um mesmo exame pode ser pedido para problemas de saúde diferentes e podem ser feitos em laboratórios diferentes.\\
Uma maneira encontrada de armazenar os dados do paciente de forma a conter seu histórico de consultas, exames e informações em geral, foi criar um prontuário que conterá um conjunto de registros sobre os mais diversos conteúdos relacionados ao paciente. 
Assim, tudo isso poderá ser visto através do prontuário do paciente que está consultando o sistema. \\
Além disso, o sistema deve possibilitar consultar quais serão as datas/horários das próximas consultas e, além disso, permitir alteração das datas (caso o paciente deseja e os médicos tenham aqueles dias disponíveis para atendimento).

\item \textbf{Gerenciamento dos funcionários}

Como funcionários o sistema terá: médicos, enfermeiros e secretários.
Eles têm cpf, rg, contato (seja telefone/ endereço), unidade(s) em que trabalham, já que o sistema integra mais de um hospital.
No caso do médico, outro atributo é a especialidade em que trabalha.

Os médicos devem poder acessar o sistema, consultar e modificar dados dos pacientes. \\
Quando ocorre uma consulta, o médico pode realizar diagnóstico, fazer prescrições e/o analisar os exames. Com base nisso, o médico receita medicamentos numa determinada dose. Os medicamentos podem trazer uma melhoria no quadro do paciente ou não surtirem muito efeito. Isso deve ser registrado, pois pode haver troca do medicamento. Assim, os médicos devem adicionar registros ao prontuário do paciente quando houver alguma consulta. \\


Os enfermeiros podem apenas consultar informações dos pacientes em que atende. \\
Os secretários são responsáveis pela parte de agendamento das consultas.

\end{itemize}



\subsection{Requisitos Funcionais}

<Diagrama de Caso de Uso para cada ator envolvido>

<Descrição de todos os casos de uso para cada ator>


\subsection{Requisitos não-funcionais}

\begin{itemize}

\item \textbf{Interface Amigável:} 

\textbf{Descrição:} O sistema deve apresentar uma interface intuitiva para os usuários do sistema, com menus e botões que facilitem a navegabilidade pelo sistema.
Como será implementado no Brasil, é interessante fazer em língua portuguesa, porque nem todos os usuários sabem inglês, por exemplo.
Além disso, ser user-friendly é importante, visto que o sistema não pode exigir conhecimentos avançados de informática. A integração de hospitais visa atender o público em geral e, portanto, tanto o vocabulário do site, quanto o modo de uso, não deve ser complexo. 

\textbf{Prioridade:} importante


\item \textbf{Sistema de Ajuda} 

\textbf{Descrição:} Um menu de ajuda é sempre interessante para que os usuários possam tirar suas dúvidas a respeito do uso do sistema, ou pelo menos uma página de FAQ. Contudo, o projeto é grande e essa funcionalidade pode ser deixada para futuras versões. 

\textbf{Prioridade:} Desejável \newline


\item \textbf{Restrições de acesso/modificação do sistema}

\textbf{Descrição:} Realiza o controle de acesso, restrições de que pode alterar os dados do sistema e está relacionado com a segurança da informação.

\textbf{Prioridade:} importante

\end{itemize}

\section{Apêndices}


\end{document}